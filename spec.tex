%&program=xelatex
%&encoding=UTF-8 Unicode
%
% Copyright © 2017 Arqadium. All rights reserved.
%
% This document contains proprietary information of ARQADIUM and/or its
% licensed developers and are protected by national and international
% copyright laws. They may not be disclosed to third parties or copied or
% duplicated in any form, in whole or in part, without the prior written
% consent of Arqadium.
%

\documentclass[12pt,english]{article}
\usepackage[a4paper,bindingoffset=0.2in,margin=1cm,footskip=0.75cm]{geometry}

\widowpenalties 1 10000
\raggedbottom

\setcounter{tocdepth}{3}

\usepackage[bookmarks=true, unicode=true, hidelinks=true,
linktoc=section]{hyperref}

\usepackage{array,multirow,graphicx}
\usepackage{mathtools}
\usepackage{changepage}
\usepackage{listings}
\usepackage[usenames,dvipsnames,svgnames,table]{xcolor}

\definecolor{src-comment}{HTML}{8F8F8F}
\definecolor{src-string}{HTML}{B03E38}
\definecolor{src-quote}{HTML}{D4554E}
\definecolor{src-numeric}{HTML}{518DED}
\definecolor{src-keyword}{HTML}{3566B5}
\definecolor{src-type}{HTML}{56914E}
\definecolor{src-ident}{HTML}{90C255}
\definecolor{src-const}{HTML}{D69B33}
\definecolor{src-symbol}{HTML}{5C5C5C}

\newlength{\tindent}
\setlength{\tindent}{\parindent}
\setlength{\parindent}{0pt}
\renewcommand{\indent}{\hspace*{\tindent}}

\setlength{\parskip}{1.25em}
\setlength\tabcolsep{3mm}

\pagenumbering{arabic}

\usepackage{mathspec}
\usepackage{fontspec}
\setallmainfonts{Palatino Linotype}
\setsansfont{WenQuanYi Zen Hei}
\setmonofont{Source Code Pro}

\renewcommand{\texttt}[1]{\ttfamily{\small{#1}}\normalfont{}}

\newcommand{\hr}{\noindent\rule{185mm}{0.4pt}}

\begin{document}
\begin{center}

\vspace*{9cm}

\Large{\textbf{Arqadium presents}} \\*
\vspace*{0.25cm}
\Huge{\textbf{Wyvern Configuration File Format}} \\*
\vspace*{1cm}
\large{\textit{July 31, 2017 (UTC)}}
\end{center}

\clearpage{}

\tableofcontents{}

\clearpage{}

\section{Objectives}

\textbf{Wyvern} aims to provide a configuration file format that is supremely
flexible for user-facing configuration settings. It is also designed to
support minification for efficient transit over networks, and provide tools
to write \textit{extensible settings,} so file authors may reference other
settings and even perform string interpolation using their values.

\subsection{Sample file}

\texttt{\textcolor{src-comment}{// I walk the line.} \\*
 \\*
\textcolor{src-keyword}{section} \\*
\textcolor{src-symbol}{\{} \\*
\hspace*{1cm}\textcolor{src-ident}{myString}~~~~\textcolor{src-symbol}{=}
\textcolor{src-quote}{"}\textcolor{src-string}{Hello, I'm Johnny
Cash.}\textcolor{src-quote}{"}\textcolor{src-symbol}{;} \\*
\hspace*{1cm}\textcolor{src-ident}{mySpecial}~~~\textcolor{src-symbol}{=}
\textcolor{src-const}{NaN}\textcolor{src-symbol}{;} \\*
\hspace*{1cm}\textcolor{src-ident}{singularity} \textcolor{src-symbol}{=}
\textcolor{src-comment}{/$\star$ we are all a part of $\star$/}
\textcolor{src-const}{Infinity}\textcolor{src-symbol}{;} \\*
\hspace*{2cm} \\*
\hspace*{1cm}\textcolor{src-comment}{// Many forms of tabulation are
supported by Wyvern.} \\*
\hspace*{1cm}\textcolor{src-keyword}{jp} \\*
\hspace*{1cm}\textcolor{src-symbol}{\{} \\*
\hspace*{2cm}\textcolor{src-comment}{// It takes in Unicode from the get-go,
and in key names too.} \\*
\hspace*{2cm}\textcolor{src-type}{"}\textcolor{src-ident}{\textsf{かたかな}}\textcolor{src-type}{"}
\textcolor{src-symbol}{=}
\textcolor{src-quote}{"}\textcolor{src-string}{\textsf{カナ}}\textcolor{src-quote}{"}\textcolor{src-symbol}{;}
\\*
\hspace*{1cm}\textcolor{src-symbol}{\}} \\*
\textcolor{src-symbol}{\}} \\*
 \\*
\textcolor{src-keyword}{Section2} \\*
\textcolor{src-symbol}{\{} \\*
\hspace*{1cm}\textcolor{src-ident}{myArray} \textcolor{src-symbol}{=}
\textcolor{src-symbol}{[} \\*
\hspace*{2cm}\textcolor{src-numeric}{1}\textcolor{src-symbol}{,} \\*
\hspace*{2cm}\textcolor{src-numeric}{2}\textcolor{src-symbol}{,} \\*
\hspace*{2cm}\textcolor{src-numeric}{3} \\*
\hspace*{1cm}\textcolor{src-symbol}{];} \textcolor{src-comment}{// Yay, arrays!} \\*
 \\*
\hspace*{1cm}\textcolor{src-comment}{/+ But... /+ can it nest? +/ Yes indeed! +/} \\*
\hspace*{1cm}\textcolor{src-ident}{alaJavaScript} \textcolor{src-symbol}{=}
\textcolor{src-quote}{"}\textcolor{src-string}{Semicolons (sorta)
optional!}\textcolor{src-quote}{"} \\*
\hspace*{1cm}\textcolor{src-type}{"}\textcolor{src-ident}{Pokémon}\textcolor{src-type}{"}
\textcolor{src-symbol}{=} \textcolor{src-symbol}{[} \\*
\hspace*{2cm}\textcolor{src-quote}{'}\textcolor{src-string}{hot}\textcolor{src-quote}{'}\textcolor{src-symbol}{,}
\\*
\hspace*{2cm}\textcolor{src-quote}{'}\textcolor{src-string}{Skitty}\textcolor{src-quote}{'}\textcolor{src-symbol}{,}
\\*
\hspace*{2cm}\textcolor{src-quote}{'}\textcolor{src-string}{on}\textcolor{src-quote}{'}\textcolor{src-symbol}{,}
\\*
\hspace*{2cm}\textcolor{src-quote}{'}\textcolor{src-string}{Wailord}\textcolor{src-quote}{'}\textcolor{src-symbol}{,}
\\*
\hspace*{2cm}\textcolor{src-quote}{'}\textcolor{src-string}{action}\textcolor{src-quote}{'}
\\*
\hspace*{1cm}\textcolor{src-symbol}{]} \\*
\hspace*{1cm}\textcolor{src-ident}{katakana} \textcolor{src-symbol}{=}
\textcolor{src-type}{\$\{section.jp:\textsf{かたかな}\}}
\textasciitilde{}
\textcolor{src-quote}{"}\textcolor{src-string}{\textsf{ カナ}}\textcolor{src-quote}{"}\textcolor{src-symbol}{;}
\textcolor{src-comment}{// katakana == "\textsf{カナ カナ}"} \\*
\hspace*{1cm}\textcolor{src-ident}{iNverse}~~\textcolor{src-symbol}{=}
\textcolor{src-symbol}{-}\textcolor{src-type}{\$\{section:singularity\}}
\textcolor{src-comment}{// iNverse == -Infinity} \\*
\textcolor{src-symbol}{\}}
}

\section{Lexical}

The lexical analysis is independent of the syntax parsing and the semantic
analysis. The lexical analyzer splits the source text up into tokens. The
lexical grammar describes the syntax of those tokens.

\subsection{Source text}

Wyvern configuration files can be in one of the following formats:

\begin{itemize}
\item UTF-8
\item UTF-16BE
\item UTF-16LE
\item UTF-32BE
\item UTF-32LE
\end{itemize}

UTF-8 is a superset of traditional 7-bit ASCII. One of the following UTF BOMs
(Byte Order Marks) can be present at the beginning of the source text:

\bgroup
\def\arraystretch{1.15}
\begin{center}
\begin{tabular}{c|l} \hline
UTF-8 & \texttt{EF BB BF} \\
UTF-16BE & \texttt{FE FF} \\
UTF-16LE & \texttt{FF FE} \\
UTF-32BE & \texttt{00 00 FE FF} \\
UTF-32LE & \texttt{FF FE 00 00} \\
\end{tabular}
\end{center}
\egroup

The source text is decoded from its source representation into Unicode
Characters. The Characters are further divided into: Whitespace, End-of-Line,
Comments, Special-Token-Sequences, Tokens, all followed by End-of-File.

The source text is split into tokens using the maximal munch technique, i.e.,
the lexical analyzer tries to make the longest token it can. For example
\texttt{>>} is a right shift token, not two greater than tokens.

\subsection{End of File}

\texttt{EndOfFile: \\*
\hspace*{1cm}\textit{physical end of the file} \\*
\hspace*{1cm}\textit{\textbf{U+0000}} \\*
\hspace*{1cm}\textit{\textbf{U+001A}} \\*}

\subsection{End of Line}

\texttt{EndOfLine: \\*
\hspace*{1cm}\textit{\textbf{U+000A}} \\*
\hspace*{1cm}\textit{\textbf{U+000D}} \\*
\hspace*{1cm}\textit{\textbf{U+000D}} \textit{\textbf{U+000A}} \\*
\hspace*{1cm}\textit{\textbf{U+2028}} \\*
\hspace*{1cm}\textit{\textbf{U+2029}} \\*
\hspace*{1cm}\textit{EndOfFile}}

Lines may be spliced by having a backslash character (\textbf{U+005C}) (and
optionally Whitespace) precede the End-of-Line character; in this case the
End-of-Line character and any Whitespace between it and the backslash is
ignored and the next line is treated as if it were part of the first.

\subsection{Whitespace}

\texttt{Whitespace: \\*
\hspace*{1cm}\textit{\textbf{U+0009}} \\*
\hspace*{1cm}\textit{\textbf{U+000B}} \\*
\hspace*{1cm}\textit{\textbf{U+000C}} \\*
\hspace*{1cm}\textit{\textbf{U+0020}} \\*}

\end{document}
