%&program=xelatex
%&encoding=UTF-8 Unicode

\documentclass[12pt,english]{article}
\usepackage[a4paper,bindingoffset=0.2in,margin=1cm,footskip=0.75cm]{geometry}

\widowpenalties 1 10000
\raggedbottom

\setcounter{tocdepth}{3}

\usepackage[bookmarks=true, unicode=true, hidelinks=true,
linktoc=section]{hyperref}

\usepackage{array,multirow,graphicx}
\usepackage{mathtools}
\usepackage{changepage}
\usepackage{listings}
\usepackage[usenames, dvipsnames]{color}

\newlength{\tindent}
\setlength{\tindent}{\parindent}
\setlength{\parindent}{0pt}
\renewcommand{\indent}{\hspace*{\tindent}}

\setlength{\parskip}{1.25em}
\setlength\tabcolsep{3mm}

\pagenumbering{arabic}

\usepackage{mathspec}
\usepackage{fontspec}
\setallmainfonts{Palatino Linotype}
\setsansfont{Helvetica Neue}
\setmonofont{Source Code Pro}

\renewcommand{\texttt}[1]{\ttfamily{\small{#1}}\normalfont{}}

\newcommand{\hr}{\noindent\rule{185mm}{0.4pt}}

\begin{document}
\begin{center}

\vspace*{9cm}

\Large{\textbf{Arqadium presents}} \\*
\vspace*{0.25cm}
\Huge{\textbf{Wyvern Configuration File Format}} \\*
\vspace*{1cm}
\large{\textit{July 31, 2017 (UTC)}}
\end{center}

\clearpage{}

\tableofcontents{}

\clearpage{}

\section{Objectives}

\textbf{Wyvern} aims to provide a configuration file format that is easy for
humans to understand yet flexible for computers to parse. It also aims to be
extensible in that this specification provides means for others to add to its
versatility with additional data types defined in \textit{Wyvern Schemas.}

\section{Lexical}

The lexical analysis is independent of the syntax parsing and the semantic
analysis. The lexical analyzer splits the source text up into tokens. The
lexical grammar describes the syntax of those tokens.

\subsection{Source text}

Wyvern configuration files can be in one of the following formats:

\begin{itemize}
\item UTF-8
\item UTF-16BE
\item UTF-16LE
\item UTF-32BE
\item UTF-32LE
\end{itemize}

UTF-8 is a superset of traditional 7-bit ASCII. One of the following UTF BOMs
(Byte Order Marks) can be present at the beginning of the source text:

\bgroup
\def\arraystretch{1.15}
\begin{center}
\begin{tabular}{c|l} \hline
UTF-8 & \texttt{EF BB BF} \\
UTF-16BE & \texttt{FE FF} \\
UTF-16LE & \texttt{FF FE} \\
UTF-32BE & \texttt{00 00 FE FF} \\
UTF-32LE & \texttt{FF FE 00 00} \\
\end{tabular}
\end{center}
\egroup

The source text is decoded from its source representation into Unicode
Characters. The Characters are further divided into: Whitespace, End-of-Line,
Comments, Special-Token-Sequences, Tokens, all followed by End-of-File.

The source text is split into tokens using the maximal munch technique, i.e.,
the lexical analyzer tries to make the longest token it can. For example
\texttt{>>} is a right shift token, not two greater than tokens.

End-of-File is defined as:

\texttt{EndOfFile: \\*
\hspace*{1cm}\textit{physical end of the file} \\*
\hspace*{1cm}\textit{\textbf{U+0000}} \\*
\hspace*{1cm}\textit{\textbf{U+001A}} \\*}

\subsection{End of Line}

\texttt{EndOfLine: \\*
\hspace*{1cm}\textit{\textbf{U+000A}} \\*
\hspace*{1cm}\textit{\textbf{U+000D}} \\*
\hspace*{1cm}\textit{\textbf{U+000D}} \textit{\textbf{U+000A}} \\*
\hspace*{1cm}\textit{\textbf{U+2028}} \\*
\hspace*{1cm}\textit{\textbf{U+2029}} \\*
\hspace*{1cm}\textit{EndOfLine}}

Lines may be spliced by having a backslash character (\textbf{U+005C}) (and
optionally Whitespace) precede the End-of-Line character; in this case the
End-of-Line character and any Whitespace between it and the backslash is
ignored and the next line is treated as if it were part of the first.

\subsection{Whitespace}

\texttt{Whitespace: \\*
\hspace*{1cm}\textit{\textbf{U+0009}} \\*
\hspace*{1cm}\textit{\textbf{U+000B}} \\*
\hspace*{1cm}\textit{\textbf{U+000C}} \\*
\hspace*{1cm}\textit{\textbf{U+0020}} \\*}

\subsection{Comments}

\texttt{Comment: \\*
\hspace*{1cm}\textit{BlockComment} \\*
\hspace*{1cm}\textit{LineComment} \\*
\hspace*{1cm}\textit{NestingBlockComment} \\*}

\texttt{BlockComment: \\*
\hspace*{1cm}\textbf{/*} \textit{Characters} \textbf{*/} \\*}

\texttt{LineComment: \\*
\hspace*{1cm}\textbf{//} \textit{Characters} \textit{EndOfLine} \\*}

\texttt{NestingBlockComment: \\*
\hspace*{1cm}\textbf{/+} \textit{NestingBlockCommentCharacters} \textbf{+/} \\*}

\texttt{NestingBlockCommentCharacters: \\*
\hspace*{1cm}\textit{NestingBlockCommentCharacter} \\*
\hspace*{1cm}\textit{NestingBlockCommentCharacter}
\textit{NestingBlockCommentCharacters} \\*}

\texttt{NestingBlockCommentCharacter: \\*
\hspace*{1cm}\textit{Character} \\*
\hspace*{1cm}\textit{NestingBlockComment} \\*}

\texttt{Characters: \\*
\hspace*{1cm}\textit{Character} \\*
\hspace*{1cm}\textit{Character} \textit{Characters} \\*}

\end{document}
